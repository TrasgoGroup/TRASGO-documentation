\chapter{Introduction}

%\addcontentsline{toc}{chapter}{Introduction}

\section{The Cosmic Rays}

At present, cosmic rays with large energies cannot be detected directly, so that we must measure the products of the atmospheric cascades of particles initiated by the incident astroparticle. In general, this cascades of secondary particles are generated by the inelastic nuclear collision between cosmic rays --those astroparticles-- and the atmospheric particles. Those secondary particles continue interacting and generating other and other secondary particles until a maximum is reached, and then the shower atenuates as far as more and more particles fall below the threshold for further particle production.

\section{The Detector}

Since 2014, in the LabCAF laboratory of the Faculty of Physics of the USC, a TRASGO type detector has been installed and taking data: TRAGALDABAS (TRAsGo for the AnaLysis of the nuclear matter Decay, the Atmosphere, the earth B-field And the Solar activity), with the intention of making a joint analysis of the data taken simultaneously with TRISTAN (TRasgo para InveSTigaciones ANtárticas), separated by a distance around $1.3 \times 10^7$ m.

This TRAGALDABAS detector is made of four planes of avalanche RCPs, but at the moment only three of them are instrumented yet. Those planes of 1.2$\times$1.5 m$^2$ are placed in a range of 1.8 m high and they are made up of 120 cells each one, placed in a 30$\times$40 array. Therefore, this device has an active area of 1.2$\times$1.5 m$^2$ and covers a vertical solid angle of $\sim $ 5 sr offering a time resolution of $\sim 300$ ps and track arriving angle resolution better than 3$^\circ$. 

\section{The Data Flow}

The detector is taking data with coincidence trigger between planes, at a rate about 7 million of registered events per day. This analog data of coincidences is converted to digital data and it is stored, along with humidity, pressure and temperature data.

For monitoring and alerting if data it is out of expected ranges, we use a software called Nagios. It is a software that provides great versatility to consult any parameter of interest in the system. The alerts generated are received by the corresponding managers (among other means) by email, when these parameters exceed the margins defined by the network administrator.

To format the numerical data and visualize it, we use Grafana. It is a platform without ani dependency and allows creating dashboards and graphs from multiple sources.

Both applications are multi-platform open-sources, licensed under the terms of the GNU General Public License and they are accessible from the computer called Trucha

\subsection{A PC Called Trucha}

It's name cames from the trout, that is a fish. In the LabCAF, the PCs (Pe-Ce-s in spanish, fishes in english) tower computers take names of fishes.

Actually, in this PC are stored the Nagios' warnings and alerts, and defined their ranges of activation. It looks like the directory tree of the Figure \ref{fg:directoryTree}.

To keep the code clean, readable, and manageable, each of the scripts in \texttt{/etc/nagios/scripts/} whose name begins with \texttt{sensor} parses the data from a single detector plane. Scripts that their name start with \texttt{check} call the classes defined in the previous ones for each of the functional detector planes.

Scripts in \texttt{/etc/nagios/objects/} are the configurations of the variables used for calling the later mentioned python scripts and where the limits of the alerts for Nagios are defined.



%\begin{figure}
%\begin{forest}
%  pic dir tree,
%  pic root,
%  for tree={directory,},
%	[/etc/nagios/
%		[...]
%		[..., file]
%		[scripts/
%			[check\_allAmbient.py, file]
%			[check\_allPower.py, file]
%			[check\_all.py, file]
%			[sensorHighVoltage\_tragaldabas.py, file]
%			[sensorHumidity\_tragaldabas.py, file]
%			[sensorPressure\_tragaldabas.py, file]
%			[sensorTemperature\_tragaldabas.py, file]
%		]
%		[objects/
%			[commands.cfg, file]
%			[contacts.cfg, file]
%			[hosts.cfg, file]
%			[printer.cfg, file]
%			[sevices.cfg, file]
%			[checkcommands.cfg, file]
%		]
%	]
%\end{forest}
%\label{fg:directoryTree}
%\caption{Directory tree of Trucha, where the programs for flow control are stored.}
%\end{figure}

...

\begin{figure}
\begin{forest}
  pic dir tree,
  pic root,
  for tree={directory,},
	[/media/Datos2TB/
		[tragaldabas/
			[data/
				[done/
				[... tryydddhhmmss.hld, file]
				]
				[Done -> done/]
				[matlab
				[... tryydddhhmmss\_TRBi.mat, file]
				]
				[sync
				[... tryydddhhmmss\_sync\_TRBi.mat, file]
				]		
			]
			[dst/]
			[external/]
			[git/]
			[historical\_data/]
			[luptab/]
			[scripts/]
		]
		[users]
		[...]
	]
\end{forest}
\label{fg:Datos2TBTree}
\caption{Directory tree of Trucha, inside the HDD with 2 TB of storage.}
\end{figure}


