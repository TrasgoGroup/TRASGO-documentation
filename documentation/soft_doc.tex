\chapter{Software Documentation}


\section{TRUFA reposiroty}

You can clone this reposiroty from \textsf{GitHub}:
\begin{lstlisting}[style=customsh]
cd <work_dir>
git clone https://github.com/TrasgoGroup/TRUFA.git
\end{lstlisting}
inside your Tragaldabas working directory directly, but forking it is recommended to make your own modifications to this code.

\section{How to fork}

If you don't have a \textsf{GitHub} account, \href{https://github.com/join}{join} first. Then, navigate to \url{https://github.com/TrasgoGroup/TRUFA.git} and fork the repository to you account by clicking in the \textsf{Fork} button, shown in Figure \ref{fig:Fork}.

Once you have forked it, you can navigate to your \textsf{GitHub} profile and go to your repositories. If you open the \textsf{TRUFA} repository you will see something similar to Figure \ref{fig:Fork}, then you are able to clone your forked repository by clicking in the green button \textsf{Code} and copying the \textsc{https} link.

You can clone this reposiroty from your \textsf{GitHub} account:
\begin{lstlisting}[style=customsh]
cd <work_dir>
git clone <copied_link>
\end{lstlisting}
inside your Tragaldabas working directory directly. Now you can edit, \texttt{git add}, \texttt{git commit}, \texttt{git push} and all you want in your new \textsf{TRUFA} directory.


\begin{figure}% [htp]
    \centering
    \includegraphics[width=\linewidth]{how_to_fork.png}
    \caption{\textsf{Fork} button is in the top-right margin of the page.}
    \label{fig:Fork}
\end{figure}



\section{The TRUFA directory}

Inside your \textsf{TRUFA} directory we need to create the lists with \texttt{output\_files.hld} from \textsc{tragaldabas} detector. Those are text files with all \texttt{hld} names separated by line-breaks, which define what files will be executed. The \textsf{python} script called \texttt{multiThreadRun.py} is the one that orders the processes to be executed in parallel for all \texttt{hld} files.

The classes explained below are defined in the files written in C that make up this directory

\section{Classes definitions}

\subsection{TRpcHit}

TRpcHit is the class which does \textit{... that things ...}, and its source code is in the trpchit.cc and trpchit.h files. It has some protected variables:

\begin{itemize}
    \item Int\_t $fTrbnum$: Number of the TRB. Each plane has a TRB.
    \item Int\_t $fCell$: 
    \item Int\_t $fCol$: 
    \item Int\_t $fRow$: 
    \item Float\_t $fX$: X coordinate.
    \item Float\_t $fY$: Y coordinate.
    \item Float\_t $fZ$: Z coordinate.
    \item Float\_t $fTime$: Track time
    \item Float\_t $fCharge$: \textbf{Absolute} (¿?) value of the charge.
\end{itemize}

The methods of TRpcSaeta class are:

\subsubsection{getCell}

\begin{lstlisting}[style=customc]
Int_t getCell();
\end{lstlisting}

It returns the protected variable $fCell$.

\[\]

\subsubsection{getCharge}

\begin{lstlisting}[style=customc]
Int_t getCharge();
\end{lstlisting}

It returns the protected variable $fCharge$.

\[\]

\subsubsection{getCol}

\begin{lstlisting}[style=customc]
Int_t getCol();
\end{lstlisting}

It returns the protected variable $fCol$.

\[\]

\subsubsection{getHit}

\begin{lstlisting}[style=customc]
void getHit(Int_t& trbnum, Int_t& cell,Int_t&  col, Int_t&  row, Float_t& x, Float_t& y, Float_t& z, Float_t& time, Float_t& charge);
\end{lstlisting}

It sets the value of the protected variables passed as parameters to the respective variables stored in memory for the detector with two planes.

\begin{itemize}
	\item Int\_t $fTrbnum \rightarrow$ Int\_t $trbnum$
	\item Int\_t $fCell \rightarrow$ Int\_t $cell$
	\item Int\_t $fCol \rightarrow$ Int\_t $col$
	\item Int\_t $fRow \rightarrow$ Int\_t $row$
	\item Float\_t $fX \rightarrow$ Float\_t $x$
	\item Float\_t $fY \rightarrow$ Float\_t $y$
	\item Float\_t $fZ \rightarrow$ Float\_t $z$
	\item Float\_t $fTime \rightarrow$ Float\_t $time$
	\item Float\_t $fCharge \rightarrow$ Float\_t $charge$
\end{itemize}



\subsubsection{getRow}

\begin{lstlisting}[style=customc]
Int_t getRow();
\end{lstlisting}

It returns the protected variable $fRow$.

\[\]

\subsubsection{getTime}

\begin{lstlisting}[style=customc]
Int_t getTime();
\end{lstlisting}

It returns the protected variable $fTime$.

\[\]

\subsubsection{getTrbnum}

\begin{lstlisting}[style=customc]
Int_t getTrbnum();
\end{lstlisting}

It returns the protected variable $fTrbnum$.

\[\]

\subsubsection{getX}

\begin{lstlisting}[style=customc]
Int_t getX();
\end{lstlisting}

It returns the protected variable $fX$.

\[\]

\subsubsection{getY}

\begin{lstlisting}[style=customc]
Int_t getY();
\end{lstlisting}

It returns the protected variable $fY$.

\[\]

\subsubsection{getZ}

\begin{lstlisting}[style=customc]
Int_t getZ();
\end{lstlisting}

It returns the protected variable $fZ$.

\[\]

\subsubsection{setCell}

\begin{lstlisting}[style=customc]
void setCell(Int_t num );
\end{lstlisting}

It sets the value passed to the $num$ parameter to the protected variable $fCell$.

\[\]

\subsubsection{setCharge}

\begin{lstlisting}[style=customc]
void setCharge(Float_t val );
\end{lstlisting}

It sets the value passed to the $val$ parameter to the protected variable $fCharge$.

\[\]

\subsubsection{setCol}

\begin{lstlisting}[style=customc]
void setCol(Int_t num );
\end{lstlisting}

It sets the value passed to the $num$ parameter to the protected variable $fCol$.

\[\]

\subsubsection{setHit}

\begin{lstlisting}[style=customc]
void setHit(Int_t trbnum,Int_t  cell,Int_t  col,Int_t  row, Float_t x, Float_t y, Float_t z, Float_t time, Float_t charge);
\end{lstlisting}

It sets the value of the parameters to their respective protected variables.

\begin{itemize}
	\item Int\_t $trbnum \rightarrow$ Int\_t $fTrbnum$
	\item Int\_t $cell \rightarrow$ Int\_t $fCell$
	\item Int\_t $col \rightarrow$ Int\_t $fCol$
	\item Int\_t $row \rightarrow$ Int\_t $fRow$
	\item Float\_t $x \rightarrow$ Float\_t $fX$
	\item Float\_t $y \rightarrow$ Float\_t $fY$
	\item Float\_t $z \rightarrow$ Float\_t $fZ$
	\item Float\_t $time \rightarrow$ Float\_t $fTime$
	\item Float\_t $charge \rightarrow$ Float\_t $fCharge$
\end{itemize}

\subsubsection{setRow}

\begin{lstlisting}[style=customc]
void setRow(Int_t num );
\end{lstlisting}

It sets the value passed to the $num$ parameter to the protected variable $fRow$.

\[\]

\subsubsection{setTime}

\begin{lstlisting}[style=customc]
void setTime(Float_t val );
\end{lstlisting}

It sets the value passed to the $val$ parameter to the protected variable $fTime$.

\[\]

\subsubsection{setTrbnum}

\begin{lstlisting}[style=customc]
void setTrbnum(Int_t num );
\end{lstlisting}

It sets the value passed to the $num$ parameter to the protected variable $fTrbnum$.

\[\]

\subsubsection{setX}

\begin{lstlisting}[style=customc]
void setX(Float_t val );
\end{lstlisting}

It sets the value passed to the $val$ parameter to the protected variable $fX$.

\[\]

\subsubsection{setY}

\begin{lstlisting}[style=customc]
void setY(Float_t val );
\end{lstlisting}

It sets the value passed to the $val$ parameter to the protected variable $fY$.

\[\]

\subsubsection{setZ}

\begin{lstlisting}[style=customc]
void setZ(Float_t val );
\end{lstlisting}

It sets the value passed to the $val$ parameter to the protected variable $fZ$.

\[\]

\subsection{TRpcHitF}

TRpcHitF is the class which does \textit{... that things ...}, and its source code is in the trpchitf.cc and trpchitf.h files. It has some private variables:

\begin{itemize}
    \item TRpcCalPar *$fPar$: 
    \item TClonesArray *$fRpcRawHits$: 
    \item TClonesArray *$fRpcHitHits$: 
    \item TActiveCells *$fActiveCells$: 
    \item Int\_t $totalNHits$: 
\end{itemize}

The methods of TRpcHitF class are:

\subsubsection{addRpcHit}

\begin{lstlisting}[style=customc]
TRpcHit* addRpcHit();
\end{lstlisting}

It returns the pointer to a new \texttt{(hits[totalNHits++]) TRpcHit()}.

\[\]

\subsubsection{execute}

IDK [...]

\[\]

\subsubsection{init}

\begin{lstlisting}[style=customc]
Int_t init(TString filename, TString filenameActive);
\end{lstlisting}

It creates a new TRpcCalPar object called $fPar$ with name $filename$ and another new TActiveCells object called $fActiveCells$ with name $filenameActive$. Then takes \texttt{gEvent -> getRpcRawHits()} and stores it in the private variable $fRpcRawHits$.

\[\]

\subsubsection{getRpcHits}

\begin{lstlisting}[style=customc]
TClonesArray*  getRpcHits();
\end{lstlisting}

It returns the private variable $fRpcHitHits$.

\[\]

\subsubsection{getRpcRawHits}

\begin{lstlisting}[style=customc]
TClonesArray*  getRpcRawHits();
\end{lstlisting}

It returns \texttt{gEvent->getRpcRawHits()}.

\[\]

\subsection{TRpcSaeta}

TRpcSaeta is the class which does \textit{... that things ...}, and its source code is in the trpcsaeta.cc and trpcsaeta.h files. All its initial values are set as $-1$ with the constructor. It has some protected variables:

\begin{itemize}
	\item Float\_t $fX$: X coordinate.
	\item Float\_t $fXP$: X slope.
	\item Float\_t $fY$: Y coordinate.
	\item Float\_t $fYP$: Y slope.
	\item Float\_t $fZ$: Z coordinate.
	\item Float\_t $fTime$: Track time.
	\item Float\_t $fSlow$: Slowness.
	\item Float\_t $fAl$: Alpha angle.
	\item Float\_t $fBe$: Beta angle.
	\item Float\_t $fGa$: Gamma angle.
	\item Int\_t   $fSaN$: Saeta order.
	\item Int\_t   $find0$: Hit index.
	\item Int\_t   $find1$: Hit index.
	\item Int\_t   $find2$: Hit index.
	\item Float\_t $fChi2$: Chi-square.
\end{itemize}

The methods of TRpcSaeta class are:

\subsubsection{getAl}

\begin{lstlisting}[style=customc]
Float_t getAl();
\end{lstlisting}

It returns the alpha $\alpha$ angle in radians $fAl$. This is the angle between the trayectory of the incident particle and the $x$-axis.

\[\]

\subsubsection{getBe}

\begin{lstlisting}[style=customc]
Float_t getBe();
\end{lstlisting}

It returns the beta $\beta$ angle in radians $fBe$. This is the angle between the trayectory of the incident particle and the $y$-axis.

\[\]

\subsubsection{getChi2}

\begin{lstlisting}[style=customc]
Float_t getChi2();
\end{lstlisting}

It returns the chi squared $\chi^2$ value $fChi2$.

\[\]

\subsubsection{getGa}

\begin{lstlisting}[style=customc]
Float_t getGa();
\end{lstlisting}

It returns the gamma $\gamma$ angle in radians $fGa$w. This is the angle between the trayectory of the incident particle and the $z$-axis.

\[\]

\subsubsection{getInd}

\begin{lstlisting}[style=customc]
Int_t getInd(Int_t n);
\end{lstlisting}

It returns the hit index for the $n$ parameter:
\begin{itemize}
	\item $n = 0 \rightarrow find0$
	\item $n = 1 \rightarrow find1$
	\item $n = 2 \rightarrow find2$
\end{itemize}

If $n \neq 0, 1, 2$, it returns $-1$.

\[\]

\subsubsection{getPhi}

\begin{lstlisting}[style=customc]
Float_t getPhi();
\end{lstlisting}

It returns the phi $\phi$ angle. This is the principal value of the arc tangent of $fAl/fBe$, expressed in radians.

\[\]

\subsubsection{getPhiDeg}

\begin{lstlisting}[style=customc]
Float_t getPhiDeg();
\end{lstlisting}

It returns the phi $\phi$ angle in degrees. This is the principal value of the arc tangent of $fAl/fBe$.

\[\]

\subsubsection{getRpcSaeta2Planes}

\begin{lstlisting}[style=customc]
void getRpcSaeta2Planes(Float_t& x0,Float_t& y0,Float_t& t0, Float_t& al,Float_t& be,Float_t& ga,Int_t& ind0, Int_t& ind1);
\end{lstlisting}

It sets the value of the protected variables passed as parameters to the respective variables stored in memory for the detector with two planes.

\begin{itemize}
	\item Float\_t $fX0 \rightarrow$ Float\_t $x0$
	\item Float\_t $fY0 \rightarrow$ Float\_t $y0$
	\item Float\_t $fTime \rightarrow$ Float\_t $t0$
	\item Float\_t $fAl \rightarrow$ Float\_t $al$
	\item Float\_t $fBe \rightarrow$ Float\_t $be$
	\item Float\_t $fGa \rightarrow$ Float\_t $ga$
	\item Int\_t $find0 \rightarrow$ Int\_t $ind0$
	\item Int\_t $find1 \rightarrow$ Int\_t $ind1$
\end{itemize}

\subsubsection{getRpcSaeta3Planes}

\begin{lstlisting}[style=customc]
void getRpcSaeta3Planes(Float_t& x0,Float_t& xP,Float_t& y0, Float_t& yP,Float_t& z, Float_t& t0,Float_t& sl, Float_t& al, Float_t& be,Float_t& ga, Int_t& san, Int_t& ind0,Int_t& ind1, Int_t& ind2, Float_t& chi2);
\end{lstlisting}

It sets the value of the protected variables passed as parameters to the respective variables stored in memory for the detector with three planes.

\begin{itemize}
	\item Float\_t $fX0 \rightarrow$ Float\_t $x0$
	\item Float\_t $fXP \rightarrow$ Float\_t $xP$
	\item Float\_t $fY0 \rightarrow$ Float\_t $y0$
	\item Float\_t $fYP \rightarrow$ Float\_t $yP$
	\item Float\_t $fZ \rightarrow$ Float\_t $z$
	\item Float\_t $fTime \rightarrow$ Float\_t $t0$
	\item Float\_t $fSlow \rightarrow$ Float\_t $sl$
	\item Float\_t $fAl \rightarrow$ Float\_t $al$
	\item Float\_t $fBe \rightarrow$ Float\_t $b$
	\item Float\_t $fGa \rightarrow$ Float\_t $ga$
	\item Float\_t $fSaN \rightarrow$ Float\_t $san$
	\item Int\_t $find0 \rightarrow$ Int\_t $ind0$
	\item Int\_t $find1 \rightarrow$ Int\_t $ind1$
	\item Int\_t $find2 \rightarrow$ Int\_t $ind2$
	\item Int\_t $fChi2 \rightarrow$ Int\_t $chi2$
\end{itemize}

\subsubsection{getSaetaN}

\begin{lstlisting}[style=customc]
Float_t getSaetaN();
\end{lstlisting}

It returns the value of the saeta order $fSaN$.

\[\]

\subsubsection{getSlow}

\begin{lstlisting}[style=customc]
Float_t getX0();
\end{lstlisting}

It returns the slownes of the particle $fSlow$, which is the inverse of velocity $1/v$ (Units?).

\[\]

\subsubsection{getTheta}

\begin{lstlisting}[style=customc]
Float_t getTheta();
\end{lstlisting}

It returns the theta $\theta$ angle. This is the principal value of the arc cosine of $fGa$, expressed in radians.

\[\]

\subsubsection{getThetaDeg}

\begin{lstlisting}[style=customc]
Float_t getThetaDeg();
\end{lstlisting}

It returns the theta $\theta$ angle in degrees. This is the principal value of the arc cosine of $fGa$.

\[\]

\subsubsection{getTime}

\begin{lstlisting}[style=customc]
Float_t getTime();
\end{lstlisting}

It returns the time of the hit measured since the tracking started (Units?) $fTime$.

\[\]


\subsubsection{getX0}

\begin{lstlisting}[style=customc]
Float_t getX0();
\end{lstlisting}

It returns the X coorinate, $fX$ (Units?).

\[\]

\subsubsection{getXP}

\begin{lstlisting}[style=customc]
Float_t getXP();
\end{lstlisting}

It returns the X slope, $fXP$ (Units?).

\[\]

\subsubsection{getY0}

\begin{lstlisting}[style=customc]
Float_t getY0();
\end{lstlisting}

It returns the Y coorinate, $fY$ (Units?).

\[\]

\subsubsection{getYP}

\begin{lstlisting}[style=customc]
Float_t getYP();
\end{lstlisting}

It returns the Y slope, $fYP$ (Units?).

\[\]

\subsubsection{getZ0}

\begin{lstlisting}[style=customc]
Float_t getZ0();
\end{lstlisting}

It returns the Z coorinate, $fZ$ (Units?).

\[\]

\subsubsection{setRpcSaeta2Planes}

\begin{lstlisting}[style=customc]
void setRpcSaeta2Planes(Float_t x0,Float_t y0,Float_t t0, Float_t al,Float_t be,Float_t ga,Int_t ind0,Int_t ind1);
\end{lstlisting}

It sets the value of the parameters to their respective protected variables for the detector with only two planes.

\begin{itemize}
	\item Float\_t $x0 \rightarrow$ Float\_t $fX0$
	\item Float\_t $y0 \rightarrow$ Float\_t $fY0$
	\item Float\_t $t0 \rightarrow$ Float\_t $fTime$
	\item Float\_t $al \rightarrow$ Float\_t $fAl$
	\item Float\_t $be \rightarrow$ Float\_t $fBe$
	\item Float\_t $ga \rightarrow$ Float\_t $fGa$
	\item Int\_t $ind0 \rightarrow$ Int\_t $find0$
	\item Int\_t $ind1 \rightarrow$ Int\_t $find1$
\end{itemize}

\subsubsection{setRpcSaeta3Planes}

\begin{lstlisting}[style=customc]
void setRpcSaeta3Planes(Float_t x0,Float_t xP,Float_t y0, Float_t yP,Float_t z, Float_t t0,Float_t sl, Float_t al, Float_t be,Float_t ga, Int_t san,Int_t ind0,Int_t ind1, Int_t ind2, Float_t chi2);
\end{lstlisting}

It sets the value of the parameters to their respective protected variables for the detector with three planes.

\begin{itemize}
	\item Float\_t $x0 \rightarrow$ Float\_t $fX0$
	\item Float\_t $xP \rightarrow$ Float\_t $fXP$
	\item Float\_t $y0 \rightarrow$ Float\_t $fY0$
	\item Float\_t $yP \rightarrow$ Float\_t $fYP$
	\item Float\_t $z \rightarrow$ Float\_t $fZ$
	\item Float\_t $t0 \rightarrow$ Float\_t $fTime$
	\item Float\_t $sl \rightarrow$ Float\_t $fSlow$
	\item Float\_t $al \rightarrow$ Float\_t $fAl$
	\item Float\_t $be \rightarrow$ Float\_t $fBe$
	\item Float\_t $ga \rightarrow$ Float\_t $fGa$
	\item Float\_t $san \rightarrow$ Float\_t $fSaN$
	\item Int\_t $ind0 \rightarrow$ Int\_t $find0$
	\item Int\_t $ind1 \rightarrow$ Int\_t $find1$
	\item Int\_t $ind2 \rightarrow$ Int\_t $find2$
	\item Int\_t $chi2 \rightarrow$ Int\_t $fChi2$
\end{itemize}

\subsubsection{setTime}

\begin{lstlisting}[style=customc]
void setTime(Float_t val);
\end{lstlisting}

It sets the value of the parameter $val$ to the protected variable $fTime$.

\[\]




\subsection{TRpcSaetaF}

TRpcSaetaF is the class which does \textit{... that things ...}, and its source code is in the trpcsaetaf.cc and trpcsaetaf.h files. It has some protected variables:

\begin{itemize}
	\item TClonesArray  *$fRpcHitHits$: 
	\item TClonesArray  *$fRpcHitCorr$: 
	\item TClonesArray  *$fRpcSaeta2Planes$: 
	\item TClonesArray  *$fRpcSaeta3Planes$: 
	\item Int\_t \textit{totalNHits}: 
	\item Int\_t \textit{totalNHits}2\textit{Planes}: 
	\item Int\_t \textit{totalNHits}3\textit{Planes}: 
	\item Int\_t \textit{totalNHitsCorr}: 
\end{itemize}

And it also has other public variables:

\begin{itemize}
	\item TMatrixF $InputSaeta2Planes$: the saeta vector with next values as arguments
	
	\begin{lstlisting}[style=customc]
TMatrixF InputSaeta2Planes(Float_t x1, Float_t y1, Float_t t1, Float_t z1, Float_t x2, Float_t y2, Float_t t2, Float_t z2);
	\end{lstlisting}
	
	where $(x_1, y_1, z_1, t_1)$ are the coordinates in the first plane and $(x_2, y_2, z_2, t_2)$ in the second one
	
%	\begin{itemize}
%		\item Float\_t $x_1$: 
%		\item Float\_t $y_1$: 
%		\item Float\_t $t_1$: 
%		\item Float\_t $z_1$: 
%		\item Float\_t $x_2$: 
%		\item Float\_t $y_2$: 
%		\item Float\_t $t_2$: 
%		\item Float\_t $z_2$: 
%	\end{itemize}
	
	\begin{align*}
%	InputSaeta2Planes \equiv \vec{S} = [ (x2*z1-x1*z2)/Dz,\\
%	 (x1-x2)/Dz, \\
%	 (y2*z1-y1*z2)/Dz, \\
%	 (y1-y2)/Dz, \\
%	 (t2*z1-t1*z2)/Dz, \\
%	 (t1-t2)/Dz ],
	InputSaeta2Planes = \left(
	\begin{array}{c}
	 \dfrac{x_2 z_1-x_1 z_2}{Dz}\\
%	 (x_2 z_1-x_1 z_2)/Dz\\
	 \dfrac{x_1-x_2}{Dz} \\
%	 (x_1-x_2)/Dz \\
	 \dfrac{y_2 z_1-y_1 z_2}{Dz} \\
%	 (y_2 z_1-y_1 z_2)/Dz \\
	 \dfrac{y_1-y_2}{Dz} \\
%	 (y_1-y_2)/Dz \\
	 \dfrac{t_2 z_1-t_1 z_2}{Dz} \\
%	 (t_2 z_1-t_1 z_2)/Dz \\
	 \dfrac{t_1-t_2}{Dz}
%	 (t_1-t_2)/Dz
	\end{array} \right),
	\end{align*}
	and where Float\_t $Dz = z1 - z2$. So that,
	
	\begin{equation}
	\notag
	InputSaeta2Planes \equiv \vec{S} = (x_0, x_p, y_0, y_p, t_0, s_0)
	\end{equation}
	is the saeta vector in the parameters space represented with respect to the origin of coordinates.
	
	
	\item TMatrixF $KMatrix$: the $K$ matrix with next values as arguments
	
	\begin{lstlisting}[style=customc]
TMatrixF KMatrix(TMatrixF SIn, Float_t z);
	\end{lstlisting}
	
	where $SIn$ is the input saeta vector
	
	\begin{align*}
	SIn = \left(
	\begin{array}{c}
	x_0\\
	x_p\\
	y_0\\
	y_p\\
	t_0\\
	s_0
	\end{array} \right),
	\end{align*}
	
	and $z$ the height of the current plane. The $k$ value is
	
	\begin{equation}
	\notag
	k = \sqrt{1 + x_p^2 + y_p^2},
	\end{equation}
	
	so that, $KMatrix$ is
	
	\begin{align*}
%	KMatrix =
	\left(
	\begin{array}{cccccc}
	w_x   & w_x z & 0 & 0 & 0 & 0\\
	w_x z & w_x z^2 + s_0^2 w_t x_p^2 z^2/k^2 & 0 & s_0^2 w_t x_p^2 z^2 /k^2 & s_0 w_t x_p z/k & s_0 w_t y_p z^2 \\
	0 & 0 & w_y & w_y z & 0 & 0\\
	0 & s_0^2 w_t x_p^2 z^2 /k^2 & w_y z & w_y z^2 + s_0^2 w_t x_p^2 z^2/k^2 & s_0 w_t y_p z/k & s_0 w_t y_p z^2\\
	0 & s_0 w_t x_p z/k & 0 & s_0 w_t y_p z/k & w_t & w_t k z\\
	0 & s_0 w_t y_p z^2 & 0 & s_0 w_t y_p z^2 & w_t k z & w_t k^2 z^2
	\end{array} \right)
	\end{align*}
	
	\item TMatrixF $AVector$: constructs the \textbf{measurement} (¿?) vector for a non-linear model with next values as arguments
	
	\begin{lstlisting}[style=customc]
TMatrixF AVector(TMatrixF SIn, Float_t x, Float_t y, Float_t t, Float_t z);
	\end{lstlisting}
	
	where $(x, y, z, t)$ are the measured coordinates and $SIn$ and $k$ have the same form as in TMatrixF $KMatrix$. So $AVector$ is
	
	\begin{align*}
	AVector = \left(
	\begin{array}{c}
	w_x x\\
	z \dfrac{w_x x k^2 + s_0 w_t x_p (t k+ s_0 (x_p^2 +y_p^2 )z)}{k^2}\\
	w_y y\\
	z \dfrac{w_y y k^2 + s_0 w_t y_p (t k+ s_0 (x_p^2 +y_p^2 )z)}{k^2}\\
	w_t \dfrac{t k + s_0 (x_p^2+ y_p^2)z}{k}\\
	w_t (t k + s_0 (x_p^2+ y_p^2)z) z
	\end{array} \right).
	\end{align*}
	
	
\end{itemize}

\[\]

The methods of TRpcSaetaF class are:

\subsubsection{addRpcHit}

\begin{lstlisting}[style=customc]
TRpcHit* addRpcHit();
\end{lstlisting}

It returns the pointer $rpchit$, which points to a new \texttt{(hits[totalNHitsCorr++]) TRpcHit()}, the next hit.

\[\]

\subsubsection{addRpcSaeta2Planes}

\begin{lstlisting}[style=customc]
TRpcSaeta* addRpcSaeta2Planes();
\end{lstlisting}

It returns the pointer $RpcSaeta$, which points to a new \texttt{(RpcSaeta2Planes} \texttt{[totalNHits2Planes++])} \texttt{TRpcSaeta()}, the next [...].

\[\]

\subsubsection{addRpcSaeta3Planes}

\begin{lstlisting}[style=customc]
TRpcSaeta* addRpcSaeta3Planes()
\end{lstlisting}

It returns the pointer $RpcSaeta$, which points to a new \texttt{(RpcSaeta3Planes} \texttt{[totalNHits3Planes++])} \texttt{TRpcSaeta()}, the next [...].

\[\]

\subsubsection{execute}

\begin{lstlisting}[style=customc]
Int_t execute();
\end{lstlisting}

The main function of the class. It makes the loops for finding hits in planes and stores them on Ttrees.

\[\]

\subsubsection{getRpcHits}

\begin{lstlisting}[style=customc]
TClonesArray* getRpcHits() {return gEvent->getRpcHits(); }
\end{lstlisting}

[...]

\[\]

\subsubsection{getRpcHitsCorr}

\begin{lstlisting}[style=customc]
ClonesArray* getRpcHitCorr() {return fRpcHitCorr; }
\end{lstlisting}

[...]

\[\]

\subsubsection{getRpcSaeta2Planes}

\begin{lstlisting}[style=customc]
ClonesArray* getRpcSaeta2Planes() {return fRpcSaeta2Planes; }
\end{lstlisting}

It returns the private $fRpcSaeta2Planes$ pointer.

\[\]

\subsubsection{getRpcSaeta3Planes}

\begin{lstlisting}[style=customc]
ClonesArray* getRpcSaeta3Planes() {return fRpcSaeta3Planes; }
\end{lstlisting}

It returns the private $fRpcSaeta3Planes$ pointer.

\[\]

\subsubsection{init}

\begin{lstlisting}[style=customc]
Int_t init();
\end{lstlisting}

It initializes the variables $fRpcHitCorr$, $fRpcSaeta2Planes$ and $fRpcSaeta3Planes$ if they are not defined by TClonesArray, creating the new ones.

\[\]

\subsection{TTMatrix}

TTMatrix is the class which does \textit{... that things ...}, and its source code is in the ttmatrix.cc and ttmatrix.h files. It has some protected variables:

\begin{itemize}
	\item Float\_t $fX1$: X coordinate in the first plane
	\item Float\_t $fY1$: Y coordinate in the first plane
	\item Float\_t $fT1$: time in the first plane
	\item Float\_t $fZ1$: Z coordinate in the first plane
	\item Float\_t $fX2$: X coordinate in the second plane
	\item Float\_t $fY2$: Y coordinate in the second plane
	\item Float\_t $fT2$: time in the second plane
	\item Float\_t $fZ2$: Z coordinate in the second plane
	\item Float\_t $fwx$: deviation of cell shape in x direction
	\item Float\_t $fwy$: deviation of cell shape in y direction
	\item Float\_t $fwz$: EN TEORÍA NO HAY INCERTIDUMBRE EN Z
\end{itemize}

And it also has other public variables:

\begin{itemize}
	\item TMatrixF $AVector$: is the vector which \textit{... do things ...} ¿Is it the \textbf{measurement} vector?
	\begin{lstlisting}[style=customc]
TMatrixF AVector(TMatrixF SIn, Float_t x, Float_t y, Float_t t, Float_t z);
	\end{lstlisting}

	where $SIn$ is the input saeta vector

	\begin{align*}
	SIn = \left(
	\begin{array}{c}
	x_0\\
	x_p\\
	y_0\\
	y_p\\
	t_0\\
	s_0
	\end{array} \right),
	\end{align*}
	
	and $(x, y, z, t)$ the measured coordinates and time. The $k$ value is 
	
	\begin{equation}
	\notag
	k = \sqrt{1 + x_p^2 + y_p^2},
	\end{equation}
	
	so that, $AVector$ is
	
	\begin{align*}
	AVector = \left(
	\begin{array}{c}
	w_x x\\
	z \dfrac{w_x x k^2 + s_0 w_t x_p (t k + s_0 (x_p^2 + y_p^2) z)}{k^2}\\
	w_y y\\
	z \dfrac{w_y y k^2 + s_0 w_t y_p (t k + s_0 (x_p + y_p) z)}{k^2}\\
	w_t \dfrac{t k + s_0 (x_p + y_p) z}{k}\\
	w_t z (t k + s_0 (x_p + y_p) z)
	\end{array} \right).
	\end{align*}
	
	Here the $w_x, w_y, w_z, w_t$ variables are initializated to zero all of them when $AVector$ is created.
	
	\item TMatrixF $KMatrix$: is the vector which \textit{... do things ...} ¿Is it the \textbf{measurement} vector?
	\begin{lstlisting}[style=customc]
TMatrixF KMatrix(TMatrixF SIn, Float_t z);
	\end{lstlisting}

	where $SIn$ is the input saeta vector

	\begin{align*}
	SIn = \left(
	\begin{array}{c}
	x_0\\
	x_p\\
	y_0\\
	y_p\\
	t_0\\
	s_0
	\end{array} \right),
	\end{align*}
	
	and $(x, y, z, t)$ the measured coordinates and time. The $k$ value has the same shape as in the $AVector$. So, the $KMatrix$ is
	
	\begin{align*}
%	KMatrix =
	\left(
	\begin{array}{cccccc}
	w_x   & w_x z                             & 0     & 0                        & 0               & 0\\
	w_x z & w_x z^2 + s_0^2 w_t x_p^2 z^2/k^2 & 0     & s_0^2 w_t x_p^2 z^2 /k^2 & s_0 w_t x_p z/k & s_0 w_t y_p z^2 \\
	0     & 0                                 & w_y   & w_y z                    & 0               & 0\\
	0     & s_0^2 w_t x_p^2 z^2 /k^2          & w_y z & w_y z^2 + s_0^2 w_t x_p^2 z^2/k^2 & s_0 w_t y_p z/k & s_0 w_t y_p z^2\\
	0     & s_0 w_t x_p z/k                   & 0     & s_0 w_t y_p z/k          & w_t             & w_t k z\\
	0     & s_0 w_t y_p z^2                   & 0     & s_0 w_t y_p z^2          & w_t k z         & w_t k^2 z^2
	\end{array} \right),
	\end{align*}
	
	with $w_x, w_y, w_z, w_t$ initializated as zero.
	
	\item TMatrixF \textit{InputSaeta2Planes}:  the saeta vector with next values as arguments
	
	\begin{lstlisting}[style=customc]
TMatrixF InputSaeta2Planes(Float_t x1, Float_t y1, Float_t t1, Float_t z1, Float_t x2, Float_t y2, Float_t t2, Float_t z2);
	\end{lstlisting}
	
	where $x_1, y_1, z_1, t_1$ are the coordinates in the first plane and $x_2, y_2, z_2, t_2$ in the second one. This is a function that returns
	
	\begin{align*}
	InputSaeta2Planes = \left(
	\begin{array}{c}
	 \dfrac{x_2 z_1-x_1 z_2}{z_1 - z_2}\\ \\
	 \dfrac{x_1-x_2}{z_1 - z_2} \\ \\
	 \dfrac{y_2 z_1-y_1 z_2}{z_1 - z_2} \\ \\
	 \dfrac{y_1-y_2}{z_1 - z_2} \\ \\
	 \dfrac{t_2 z_1-t_1 z_2}{z_1 - z_2} \\ \\
	 \dfrac{t_1-t_2}{z_1 - z_2}
	\end{array} \right),
	\end{align*}
	
	and assigns the parameters $x_1, y_1, z_1, t_1, x_2, y_2, z_2, t_2$ to the protected variables of the calss $fX1 , fY1, fZ1, fT1, fX2, fY2, fZ2, fT2$ respectively.
	
\end{itemize}

\subsection{Unpacker}

Unpacker is the class which does \textit{... that things ...}, and its source code is in the trpcunpacker.cc and trpcunpacker.h files.



Unpacker has some protected variables:

\begin{itemize}
	\item HldEvent* $pEvent$: current event read from file.
    \item Int\_t $EventNr$: event Counter.
    \item Int\_t $EventLimit$: maximum event number per file.
    \item Int\_t $subEvtId$:
    \item TFile* $pRootFile$: pointer to TFile with the output tree.
    \item std::string $inputFile$: wk 28.05
    \item std::string $outputFile$: wk 28.05
    \item Int\_t \textit{fpga\_code}: address of the data source (e.g. given fpga ) decoded from hld file.
    \item Int\_t $refCh$: 
\end{itemize}



\subsubsection{* The Constructors}

\begin{lstlisting}[style=customc]
Unpacker(const char* dir,const char* name, const char* odir, Int_t nEvt, TString luptab, TString calpar);
\end{lstlisting}

Using this construtor of unpacker we have access the data output of the tracking code. By executing this code into the \textsc{root} command line we can check whether or not what we rebuild is compatible with the current methods. Parameters:

\begin{itemize}
	\item const char* $dir$: Directory of input data, which is the output of Tragaldabas.
	\item const char* $name$: Name of the file to unpack, (i. e. \texttt{tr18249041152.hld}.)
	\item const char* $odir$: Output directory, here will appear the unpacked files.
	\item Int\_t $nEvt$: Number of events (i. e. 1000).
	\item TString $luptab$: Name of the luptable txt file (i. e. \texttt{luptable\_corr\_20180423.txt}).
	\item TString $calpar$: Name of the CalPar txt file (i. e. \texttt{2018\_day\_203\_CalPars.txt}).
\end{itemize}

\[\]

The methods of the Unpacker class are:

\subsubsection{eventLoop}

\begin{lstlisting}[style=customc]
Bool_t eventLoop(Int_t NbEvt=50000,Int_t startEvt=0);
\end{lstlisting}

Loop over all events, data is written to the root tree.

\begin{itemize}
	\item Int\_t $NbEvt$: 
	\item Int\_t $startEvt$: 
\end{itemize}

\subsubsection{eventLoopFillCal}

\begin{lstlisting}[style=customc]
Bool_t eventLoopFillCal(Int_t nbEvt, Int_t startEv, TH1D** hq, TH1D** hdt, TH1D** hdt2);
Bool_t eventLoopFillCal(Int_t nbEvt, Int_t startEv, TH1D** h1D, TH2D** h2D, TH3D** h3D);
\end{lstlisting}

Loop over all events, data written to the root tree

\begin{itemize}
	\item Int\_t $NbEvt$: 
	\item Int\_t $startEvt$: 
	\item TH1D** $hq$: 
	\item TH1D** $h1D$: 
	\item TH1D** $hdt$: 
	\item TH2D** $h2D$: 
	\item TH1D** $hdt2$: 
	\item TH3D** $h3D$: 
\end{itemize}

\subsubsection{eventLoopSyncCheck}

\begin{lstlisting}[style=customc]
Int_t eventLoopSyncCheck(Int_t nbEvt,Int_t startEv);
\end{lstlisting}

Loop over all events, data is written to the root tree.

\begin{itemize}
	\item Int\_t $NbEvt$: 
	\item Int\_t $startEvt$: 
\end{itemize}

\subsubsection{fillCalibration}

\begin{lstlisting}[style=customc]
void Unpacker::fillCalibration(const char* dir, TString list, const char* odir,const char* ofile, Int_t nEvt,Int_t n);
\end{lstlisting}

\begin{itemize}
	\item const char* $dir$: path to input file.
	\item TString $list$: list of \texttt{files.hld}.
	\item const char* $odir$: path to output directory.
	\item const char* $ofile$: path to output file.
	\item Int\_t $nEvt$: number of events.
	\item Int\_t $n$
	\begin{itemize}
		\item $n = 0$ (standard mode): just calculate the pedestals and exchande in the parameter file which is previously declared.
		\item $n = 1$ (special mode):  create pedestals and set time offsets to 0.
	\end{itemize}
\end{itemize}


\subsubsection{fillHistograms}

\begin{lstlisting}[style=customc]
void fillHistograms(const char* dir, TString list, const char* odir, const char* ofile, Int_t nEvt, Int_t n,Int_t n2);
\end{lstlisting}

\begin{itemize}
	\item const char* $dir$
	\item TString $list$
	\item const char* $odir$
	\item const char* $ofile$
	\item Int\_t $nEnv$
	\item Int\_t $n$
	\item Int\_t $n2$
	
\end{itemize}

\subsubsection{getFileHitFinderPar}

\begin{lstlisting}[style=customc]
void getFileHitFinderPar( void );
\end{lstlisting}

It gets the variable TString $fileHitFinderPar$, the name of the file with hit finder params.

\[\]

\subsubsection{getFileHitFinderParOut}

\begin{lstlisting}[style=customc]
void getFileHitFinderParOut( void );
\end{lstlisting}

It gets the variable TString $fileHitFinderParOut$, the name of the file with hit finder params.

\[\]

\subsubsection{getFileLookupPar}

\begin{lstlisting}[style=customc]
void getFileLookupPar( void );
\end{lstlisting}

It gets the variable TString $fileLookupPar$, the name of the file with lookup params.

\[\]

\subsubsection{getFileTrackFinderPar}

\begin{lstlisting}[style=customc]
void getFileTrackFinderPar( void );
\end{lstlisting}

It gets the variable TString $fileTrackFinderPar$, the name of the file with track finder params.

\[\]

\subsubsection{getpEvent}

\begin{lstlisting}[style=customc]
HldEvent* getpEvent(void);
\end{lstlisting}

It returns the pointer HldEvent* $pEvent$.

\[\]

\subsubsection{getEventLimit}

\begin{lstlisting}[style=customc]
Int_t getEventLimit()
\end{lstlisting}

It returns the Int\_t $EventLimit$.

\[\]

\subsubsection{getEventNr}

\begin{lstlisting}[style=customc]
Int_t getEventNr()
\end{lstlisting}

It returns the Int\_t $EventNr$.

\[\]

\subsubsection{HexStrToInt}

\begin{lstlisting}[style=customc]
UInt_t HexStrToInt(const char* str) {
	UInt_t t;
	std::stringstream s;
	s << std::hex << str;
	s >> t;
	return t;
 }
\end{lstlisting}

It is easy to see that \texttt{HexStrToInt} takes a const char* $str$ as argument and returns its integer value as UInt\_t $t$.

\[\]

\subsubsection{setInputFile}

\begin{lstlisting}[style=customc]
std::string setInputFile(const char* filename);
std::string setInputFile(const char* dir,const char* filename); 
\end{lstlisting}

It sets $filename$ as name of the \texttt{filename.hld} input file and returns a std::string \textit{inputFile} (\textit{wk 28.05}). If a directory $dir$ is passed as parameter, the location for \texttt{filename.hld} is changed to $dir$.

\[\]

\subsubsection{setFileHitFinderPar}

\begin{lstlisting}[style=customc]
void setFileHitFinderPar( TString fileName);
\end{lstlisting}

It sets $fileName$ to the variable TString $fileHitFinderPar$, the name of the file with hit finder params.

\[\]

\subsubsection{setFileHitFinderParOut}

\begin{lstlisting}[style=customc]
void setFileHitFinderParOut( TString fileName);
\end{lstlisting}

It sets $fileName$ to the variable TString $fileHitFinderParOut$, the name of the file with hit finder params.

\[\]

\subsubsection{setFileLookupPar}

\begin{lstlisting}[style=customc]
void setFileLookupPar( TString fileName);
\end{lstlisting}

It sets $fileName$ to the variable TString $fileLookupPar$, the name of the file with lookup params.

\[\]

\subsubsection{setFileTrackFinderPar}

\begin{lstlisting}[style=customc]
void setFileTrackFinderPar( TString fileName);
\end{lstlisting}

It sets $fileName$ to the variable TString $fileTrackFinderPar$, the name of the file with track finder params.

\[\]

\subsubsection{setpEvent}

\begin{lstlisting}[style=customc]
Bool_t setpEvent(Int_t subId);
void setpEvent(HldEvent* evt);
\end{lstlisting}

It sets $pEvent$ (the current event read from file) by reading hld file, where $subId$ is the subevent id and returns $kTRUE$. If parameter is a pointer, pEvent is set as $evt$ and doesn't returns anything.

\[\]

\subsubsection{setRootFile}

\begin{lstlisting}[style=customc]
Bool_t setRootFile(const char* filename);
\end{lstlisting}

It sets $filename$ as name to a new root output file and returns the \textsc{root} specific constant $kTRUE$.

\[\]


\subsubsection{syncCheck}

\begin{lstlisting}[style=customc]
Int_t syncCheck(const char* dir, TString file, Int_t nEvt,Int_t n);
\end{lstlisting}

\begin{itemize}
	\item const char* $dir$
	\item TString $file$
	\item Int\_t $nEnv$
	\item Int\_t $n$
	\begin{itemize}
		\item $n = 0$ (standard mode): just calculate the pedestals and exchande in the parameter file which is previously declared.
		\item $n = 1$ (special mode):  create pedestals and set time offsets to 0.
	\end{itemize}
	
\end{itemize}
